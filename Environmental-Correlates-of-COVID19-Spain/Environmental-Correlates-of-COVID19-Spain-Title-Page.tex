\documentclass[]{elsarticle} %review=doublespace preprint=single 5p=2 column
%%% Begin My package additions %%%%%%%%%%%%%%%%%%%
\usepackage[hyphens]{url}

  \journal{Geographical Analysis} % Sets Journal name


\usepackage{lineno} % add
\providecommand{\tightlist}{%
  \setlength{\itemsep}{0pt}\setlength{\parskip}{0pt}}

\usepackage{graphicx}
\usepackage{booktabs} % book-quality tables
%%%%%%%%%%%%%%%% end my additions to header

\usepackage[T1]{fontenc}
\usepackage{lmodern}
\usepackage{amssymb,amsmath}
\usepackage{ifxetex,ifluatex}
\usepackage{fixltx2e} % provides \textsubscript
% use upquote if available, for straight quotes in verbatim environments
\IfFileExists{upquote.sty}{\usepackage{upquote}}{}
\ifnum 0\ifxetex 1\fi\ifluatex 1\fi=0 % if pdftex
  \usepackage[utf8]{inputenc}
\else % if luatex or xelatex
  \usepackage{fontspec}
  \ifxetex
    \usepackage{xltxtra,xunicode}
  \fi
  \defaultfontfeatures{Mapping=tex-text,Scale=MatchLowercase}
  \newcommand{\euro}{€}
\fi
% use microtype if available
\IfFileExists{microtype.sty}{\usepackage{microtype}}{}
\bibliographystyle{elsarticle-harv}
\ifxetex
  \usepackage[setpagesize=false, % page size defined by xetex
              unicode=false, % unicode breaks when used with xetex
              xetex]{hyperref}
\else
  \usepackage[unicode=true]{hyperref}
\fi
\hypersetup{breaklinks=true,
            bookmarks=true,
            pdfauthor={},
            pdftitle={A spatio-temporal analysis of the environmental correlates of COVID-19 incidence in Spain},
            colorlinks=false,
            urlcolor=blue,
            linkcolor=magenta,
            pdfborder={0 0 0}}
\urlstyle{same}  % don't use monospace font for urls

\setcounter{secnumdepth}{5}
% Pandoc toggle for numbering sections (defaults to be off)


% Pandoc header



\begin{document}
\begin{frontmatter}

  \title{A spatio-temporal analysis of the environmental correlates of COVID-19
incidence in Spain}
    \author[McMaster University]{Antonio Paez\corref{1}}
   \ead{paezha@mcmaster.ca} 
    \author[Universidad Politecnica de Cartagena]{Fernando A. Lopez}
   \ead{fernando.lopez@upct.es} 
    \author[Departamento de Economia]{Tatiane Menezes}
   \ead{tatiane.menezes@ufpe.br} 
    \author[Nucleo de Pesquisa]{Renata Cavalcanti}
   \ead{renata.vcsantos@gmail.com} 
    \author[Nucleo de Pesquisa]{Maira Galdino da Rocha Pitta}
   \ead{mgrpitta@ufpe.br} 
      \address[McMaster University]{School of Geography and Earth Sciences, McMaster University, 1281 Main
St W, Hamilton, ON, L8S 4K1, Canada}
    \address[Universidad Politecnica de Cartagena]{Departamento de Metodos Cuantitativos, Ciencias Juridicas, y Lenguas
Modernas, Universidad Politecnica de Cartagena, Calle Real Numero 3,
30201, Cartagena, Murcia, Spain}
    \address[Departamento de Economia]{Departamento de Economia, Universidade Federal de Pernambuco, Av dos
Economistas, s/n - Cidade Universitária, Recife - PE, 50670-901, Brasil}
    \address[Nucleo de Pesquisa]{Núcleo de Pesquisa em Inovação Terapêutica NUPIT / UFPE, Av.
Prof.~Moraes Rego, 1235 - Cidade Universitária, Recife, PE, CEP
50670-901, Brazil}
      \cortext[1]{Corresponding Author}
  
  \begin{abstract}
  Early in 2020 the novel SARS-CoV2 swept the globe, disrupting health
  systems and the economy. Public health interventions have slowed the
  spread of the virus, albeit at a high cost. How and when to ease
  restrictions to movement hinges in part on whether SARS-CoV2 will
  display seasonality due to variations in temperature, humidity, and
  hours of sunshine. Here, we address this by means of a spatio-temporal
  analysis of the incidence of COVID-19 in Spain. Use of spatial Seemingly
  Unrelated Regressions (SUR) allows us to model the incidence of reported
  cases of the disease per 100,000 population as an interregional
  contagion process, in addition to a function of temperature, humidity,
  and sunshine. In the analysis we also control for GDP per capita,
  percentage of older adults in the population, population density, and
  presence of mass transit systems. The results support the hypothesis
  that incidence of the disease is lower at higher temperatures and higher
  levels of humidity. Sunshine, in contrast, displays a positive
  association with incidence of the disease. Our control variables also
  yield interesting insights. Higher incidence is associated with higher
  GDP per capita and presence of mass transit systems in the province; in
  contrast, population density and percentage of older adults display
  negative associations with incidence of COVID-19.
  \end{abstract}
  
 \end{frontmatter}




\end{document}


